\documentclass{article}

\usepackage[brazilian]{babel}
\usepackage[letterpaper,top=2cm,bottom=2cm,left=3cm,right=3cm,marginparwidth=1.75cm]{geometry}

\usepackage{amsmath}
\usepackage{graphicx}
\usepackage[colorlinks=true, allcolors=blue]{hyperref}

\title{Publish/Subscribe Service}
\author{Vinícius de Oliveira Campos dos Reis
\and 11041416}
\date{}

\begin{document}
    \maketitle

    \section{O projeto}\label{sec:introducao}

    O projeto é uma implementação principal em Kotlin (cliente secundário em Python) que utiliza o Gradle como compilador e gerenciador de dependências.
    A arquitetura de cada componente da aplicação é multi-módulos, seguindo ao máximo uma arquitetura limpa e orientada a domínios.

    \section{Como executar}\label{sec:como-executar}

    Todos módulos neste projeto podem ser executados através do Makefile na raiz do projeto, cada um com seu target.
    Confira abaixo como cada um deles pode ser executado:

    \begin{itemize}
        \item Servidor: \textit{make server}
        \item Cliente Kotlin: \textit{make client}
        \item Cliente Python: \textit{make py\_client}
    \end{itemize}

    Note que o servidor deve ser executado antes dos clientes para o correto funcionamento dos clientes.

    \section{Implementação}\label{sec:implementacao}

    Da forma mais direta possível, veja a função das principais implementações da aplicação (note que removemos da referência o início comum entre elas, que é o pacote \textit{io.github.vinicreis.pubsub.server}):

    \subsection{Servidor}\label{subsec:server}

    \begin{itemize}
        \item \textit{core.grpc.service.QueueServiceGrpc}: Ponto de entrada do serviço gRPC. Implementação de classes geradas pela biblioteca oficial do gRPC, que delega a essa classe o que é feito em cada operação do servidor.

        \item \textit{core.data.database.postgres.repository.QueueRepositoryDatabase}: repositório de filas ativas armazenadas num banco de dados.

        \item \textit{core.data.database.postgres.repository.TextMessageRepositoryDatabase}: repositório de mensagens pendentes armazenadas num banco de dados.

        \item \textit{core.data.database.postgres.repository.EventRepositoryDatabase}: repositórios de eventos pendentes, como a criação e remoção de filas e mensagens postadas.
        É válido destacar que, visando manter a máxima consistências, os eventos são salvos na mesma transação da operação que gera o evento em si.

        \item \textit{core.grpc.service.SubscriberManagerServiceImpl}: serviço de gerenciamento de inscrições.
        Responsável por monitorar os eventos ocorridos no repositório de eventos e gerenciar filas de acordo com seu tipo, selecionando quais inscritos devem ser notificados de dados eventos.

    \end{itemize}

    \subsection{Clientes}\label{subsec:clientes}

    Os clientes, tanto em Kotlin quanto em Python, são parecidos tanto em implementação quanto em funcionamento.
    Estes são interativos via linha de comando, executando um laço onde o usuário seleciona uma das operações para ser realizada ou encerra a aplicação.

\end{document}